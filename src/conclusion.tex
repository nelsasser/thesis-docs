\chapter{Conclusions}\label{ch:conclusion}

In this work we presented a novel methodology for joint probabilistic day-ahead energy price forecasts over a set of
geographically distributed price nodes and several practices that can be used to improve forecasting results using
recent advances in deep generative modeling.
Our proposed methodology allows for both exact conditional likelihood estimation and efficient conditional sampling for
forecasting tasks.
Comparisons of our model to both state-of-the-art open-access benchmarks in energy price forecasting literature and
commercial physics-based optimal powerflow tools provide clear evidence for the superiority of our proposed methodology
both in terms of forecast accuracy and more expressive information-rich probabilistic forecasting in comparison to
traditional point-forecasting methods.
Despite this, we believe the methodology presented in this work can be further improved with the following
lines of continued research,

\begin{enumerate}
    \item \textbf{Model Robustness and Interpretability}: A key issue faced during development of these methods was
          the use of non-standard robust metrics when tracking model fit due to the instabilities observed for mean
          NLL calculations caused by sporadic and extremely low log-likelihood evaluations.
          While workarounds were found, they do not address the root causes of these instabilities.
          We believe the application of modern robust deep learning methods to be a worthwhile investigation to determine
          if more stable results can be achieved by improving model robustness.
          Additionally, such techniques could be used to provide interpretability as to the quantities and market
          conditions to which our model is most sensitive.
          Such transparency is desirable for the utilization of these models when executing positions in the markets
          based on information provided by these forecasts.
    \item \textbf{Hierarchical Modeling and Uncertainty Propagation}: A key assumption of our methods is that the
          quantities we condition our forecasts on have no uncertainty; however, load, generation, and meteorological
          forecasts are known to be uncertain and error-prone.
          While more accurate forecasts would likely improve results, no forecast is 100\% correct, 100\% confident, 100\%
          of the time.
          Instead, capturing the uncertainty of those upstream fundamental forecasts in a separate density function
          $p(c_t)$ and constructing a hierarchical density estimate $p(x_t|c_t)p(c_t)$ may better capture the
          relationship between our fundamental forecasts and resulting price forecasts as uncertainty in fundamental
          forecasts can be propagated into the uncertainty of price forecasts.
    \item \textbf{Generalization to Other Markets and a Greater Number of Price Nodes}: Our experiments in this work
          were limited to the study of a single set of 45 price nodes in the PJM market.
          There are a total of 8 day-ahead energy markets in North America, with many more in Europe and the rest of the
          world.
          These markets each can have significant differences between one another in terms of regulation, available
          generation, load profiles, weather patterns, etc. that all have an outstanding effect on day-ahead price actions.
          Additionally, we only looked at a small set of 45 price nodes despite markets often being comprised of hundreds
          if not thousands of distinct price nodes.
          Further research into the generalizability of our proposed methods into other markets and the
          issues faced when scaling up forecasts to larger sets of price nodes are believed to be of great practical
          importance to the deployment and execution of this methodology.
\end{enumerate}
